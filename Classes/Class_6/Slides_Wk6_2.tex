% Font options: 10pm, 11pt, 12pt
% Align headings left instead of center: nocenter
\documentclass[xcolor=x11names,compress]{beamer}\usepackage[]{graphicx}\usepackage[]{color}
% maxwidth is the original width if it is less than linewidth
% otherwise use linewidth (to make sure the graphics do not exceed the margin)
\makeatletter
\def\maxwidth{ %
  \ifdim\Gin@nat@width>\linewidth
    \linewidth
  \else
    \Gin@nat@width
  \fi
}
\makeatother

\definecolor{fgcolor}{rgb}{0.345, 0.345, 0.345}
\newcommand{\hlnum}[1]{\textcolor[rgb]{0.686,0.059,0.569}{#1}}%
\newcommand{\hlstr}[1]{\textcolor[rgb]{0.192,0.494,0.8}{#1}}%
\newcommand{\hlcom}[1]{\textcolor[rgb]{0.678,0.584,0.686}{\textit{#1}}}%
\newcommand{\hlopt}[1]{\textcolor[rgb]{0,0,0}{#1}}%
\newcommand{\hlstd}[1]{\textcolor[rgb]{0.345,0.345,0.345}{#1}}%
\newcommand{\hlkwa}[1]{\textcolor[rgb]{0.161,0.373,0.58}{\textbf{#1}}}%
\newcommand{\hlkwb}[1]{\textcolor[rgb]{0.69,0.353,0.396}{#1}}%
\newcommand{\hlkwc}[1]{\textcolor[rgb]{0.333,0.667,0.333}{#1}}%
\newcommand{\hlkwd}[1]{\textcolor[rgb]{0.737,0.353,0.396}{\textbf{#1}}}%
\let\hlipl\hlkwb

\usepackage{framed}
\makeatletter
\newenvironment{kframe}{%
 \def\at@end@of@kframe{}%
 \ifinner\ifhmode%
  \def\at@end@of@kframe{\end{minipage}}%
  \begin{minipage}{\columnwidth}%
 \fi\fi%
 \def\FrameCommand##1{\hskip\@totalleftmargin \hskip-\fboxsep
 \colorbox{shadecolor}{##1}\hskip-\fboxsep
     % There is no \\@totalrightmargin, so:
     \hskip-\linewidth \hskip-\@totalleftmargin \hskip\columnwidth}%
 \MakeFramed {\advance\hsize-\width
   \@totalleftmargin\z@ \linewidth\hsize
   \@setminipage}}%
 {\par\unskip\endMakeFramed%
 \at@end@of@kframe}
\makeatother

\definecolor{shadecolor}{rgb}{.97, .97, .97}
\definecolor{messagecolor}{rgb}{0, 0, 0}
\definecolor{warningcolor}{rgb}{1, 0, 1}
\definecolor{errorcolor}{rgb}{1, 0, 0}
\newenvironment{knitrout}{}{} % an empty environment to be redefined in TeX

\usepackage{alltt}
%\documentclass[xcolor=x11names,compress,handout]{beamer}
\usepackage[]{graphicx}
\usepackage[]{color}
\usepackage{booktabs}
\usepackage{hyperref}
\usepackage{tikz}
\usepackage{multirow}
\usepackage{dcolumn}
\usepackage{bigstrut}
\usepackage{amsmath} 
\usepackage{xcolor,colortbl}
\usepackage{amssymb}
%\newcommand{\done}{\cellcolor{teal}#1}

%% Beamer Layout %%%%%%%%%%%%%%%%%%%%%%%%%%%%%%%%%%
\useoutertheme[subsection=false,shadow]{miniframes}
\useinnertheme{default}
\usefonttheme{serif}
\usepackage{Arev}
\usepackage{pdfpages}

\setbeamerfont{title like}{shape=\scshape}
\setbeamerfont{frametitle}{shape=\scshape, size=\normalsize}

\definecolor{dkblue}{RGB}{0,0,102}

\setbeamercolor*{lower separation line head}{bg=dkblue} 
\setbeamercolor*{normal text}{fg=black,bg=white} 
\setbeamercolor*{alerted text}{fg=red} 
\setbeamercolor*{example text}{fg=black} 
\setbeamercolor*{structure}{fg=black} 
 
\setbeamercolor*{palette tertiary}{fg=black,bg=black!10} 
\setbeamercolor*{palette quaternary}{fg=black,bg=black!10} 

\renewcommand{\(}{\begin{columns}}
\renewcommand{\)}{\end{columns}}
\newcommand{\<}[1]{\begin{column}{#1}}
\renewcommand{\>}{\end{column}}

\setbeamertemplate{navigation symbols}{} 
\setbeamertemplate{footline}[frame number]
\setbeamertemplate{caption}{\raggedright\insertcaption\par}

\setbeamersize{text margin left=5pt,text margin right=5pt}

%%%%%%%%%%%%%%%%%%%%%%%%%%%%%%%%%%%%%%%%%%%%%%%%%%


\title{FLS 6415 - Causal Inference for the Political Economy of Development}
\subtitle{Week 6 - Social Accountability, Information \& Instrumental Variables}
\author{Jonathan Phillips}
\date{September  2017}
\IfFileExists{upquote.sty}{\usepackage{upquote}}{}
\begin{document}

\frame{\titlepage}

\section{Causal Inference}

\begin{frame}
\frametitle{Instrumental Variables}
\begin{itemize}
\item What can we do when the treatment assignment mechanism is not 'as-if' random?
\pause
\item Natural experiments focus on a specific \textbf{part} of treatment assignment that is 'as-if' random
\pause
\item An 'instrument' is a variable which assigns treatment in an 'as-if' random way
\pause
\begin{itemize}
\item Or at least in a way which is 'exogenous' - not related to confounders
\item Even if other confounding variables \textbf{also} affect treatment
\end{itemize}
\end{itemize}
\end{frame}

\begin{frame}
\frametitle{Instrumental Variables}
\begin{itemize}
\item We can use the instrument to isolate 'as-if' random variation in treatment, and use that to estimate the effect of treatment on the outcome
\pause
\item NOT the effect of the instrument on the outcome
\end{itemize}
\end{frame}

\begin{frame}
\frametitle{Instrumental Variables}
\begin{itemize}
\item Example Instruments:
\begin{itemize}
\item Rainfall for conflict 
\item Sex-composition for effect of third child
\item Distance from the coast for exposure to slave trade
\end{itemize}
\end{itemize}
\end{frame}

\begin{frame}
\frametitle{Instrumental Variables}
\begin{itemize}
\item Instrumental Variables Assumptions
\begin{itemize}
\item \textbf{Strong First Stage:} The Instrument must \textbf{affect} the treatment
\pause
\item We can test this with a simple regression: $Treatment \sim Instrument$
\pause
\item The instrument should be a significant predictor of treatment
\item Rule-of-thumb: $F-statistic > 10$
\end{itemize}
\end{itemize}
\end{frame}

\begin{frame}
\frametitle{Instrumental Variables}
\begin{itemize}
\item Instrumental Variables Assumptions:
\begin{itemize}
\item \textbf{Exclusion Restriction:} The Instrument \textbf{ONLY} affects the outcome through its effect on treatment, and not directly
\pause
\item Formally, $cov(Instrument,\text{errors in main regression Y }\sim D)=0$
\pause
\item \textbf{We cannot test or prove this assumption!}
\pause
\item Theory and qualitative evidence needed to argue that the instrument is not correlated with any other factors affecting the outcome
\item Sometimes, the exclusion restriction may be more credible if we include controls
\end{itemize}
\end{itemize}
\end{frame}

\begin{frame}
\frametitle{Instrumental Variables}
\begin{itemize}
\item Instrumental Variables Methodology:
\pause
\begin{enumerate}
\item Use an all-in-one package, eg. \textit{ivreg} in the \textit{AER} package
\begin{itemize}
\item Specify the formula: $Y ~ D | Instrument$
\pause
\end{itemize}
\item Conduct 2-Stage Least Squares: 
\pause
\begin{itemize}
\item Isolate the variation in treatment caused by the instrument: $D \sim Instrument$
\pause
\item Save the predicted values from this regression: $\hat{D} = D \sim Instrument$
\pause
\item Estimate how the predicted values affect the outcome: $Y \sim \hat{D}$
\pause
\item Interpret the coefficient on $\hat{D}$
\end{itemize}
\end{enumerate}
\end{itemize}
\end{frame}

\begin{frame}
\frametitle{Instrumental Variables}
\begin{itemize}
\item IV Interpretation:
\pause
\begin{itemize}
\item Your coefficient is a causal estimate ONLY for units that were actually treated \textbf{because of the instrument}
\pause
\item They don't tell us about the causal effect for other units that never responded to the instrument
\pause
\item We call our causal effect estimate a 'Local Average Treatment Effect' (LATE)
\item 'Local' to the units whose treatment status actually changed
\pause
\end{itemize}
\item Remember, those 'Local' units are not representative so we can't generalize
\end{itemize}
\end{frame}

\begin{frame}
\frametitle{Instrumental Variables}
\begin{itemize}
\item Types of IV Regressions:
\pause
\begin{enumerate}
\item \textbf{Confounded Regression:} The mistaken regression: $Y \sim D$
\pause
\item \textbf{First-Stage Regression:} Checking the instrument is valid: $D \sim IV$
\pause
\item \textbf{IV Regression:} All-in-one estimate of the effect of treatment on the outcome: $Y \sim D | IV$
\pause
\item \textbf{2-Stage Least Squares:} Two linear regressions: correct coefficient, wrong p-value: $D \sim IV, Y \sim \hat{D}$
\pause
\item \textbf{Reduced-Form Regression:} Estimate of the Instrument on the Outcome, ignoring treatment mediation: $Y \sim IV$
\end{enumerate}
\end{itemize}
\end{frame}

\begin{frame}
\frametitle{Instrumental Variables}
\begin{itemize}
\item Instruments for Non-compliance
\pause
\begin{itemize}
\item With an instrument and treatment we can divide our units into four types:
\begin{table}[htbp]
  \centering
    \begin{tabular}{|p{3cm}|p{3cm}|p{3cm}|}
    \hline
    \multicolumn{1}{|p{3cm}|}{\textbf{Treatment Status if Instrument=0}} & \multicolumn{1}{p{3cm}|}{\textbf{Treatment Status if Instrument=1}} & \textbf{Unit Type} \bigstrut\\
    \hline
    0     & 1     & Complier \bigstrut\\
    \hline
    0     & 0     & Never-taker \bigstrut\\
    \hline
    1     & 1     & Always-taker \bigstrut\\
    \hline
    1     & 0     & Defier \bigstrut\\
    \hline
    \end{tabular}%
  \label{tab:addlabel}%
\end{table}%
\pause
\end{itemize}
\item LATE just means we don't learn anything about Never-takers and Always-takers from Instrumental Variables
\begin{itemize}
\item Because the instrument doesn't do anything to affect treatment for these units
\pause
\end{itemize}
\item We also need to \textbf{assume} Defiers don't exist
\item So LATE = Causal Effect for Compliers 
\end{itemize}
\end{frame}

\begin{frame}
\frametitle{Instrumental Variables}
\begin{itemize}
\item Instruments for Non-compliance in Experiments
\pause
\begin{itemize}
\item Normally we analyze experiments based on randomized treatment
\item But what if \textbf{assignment} to treatment is different from \textbf{taking} the treatment?
\begin{itemize}
\item Eg. If government implementation failed in some places
\end{itemize}
\end{itemize}
\end{itemize}
\end{frame}

\begin{frame}
\frametitle{Instrumental Variables}
\begin{itemize}
\item Instruments for Non-compliance in Experiments
\pause
\begin{itemize}
\item We can still use randomization as an instrument for treatment
\pause
\item The causal effect estimate of our experiment is now LATE
\begin{itemize}
\item These estimates are \textbf{internally valid} for compliers
\item But they are NOT \textbf{externally valid} for non-compliers
\item Since whether you accepted treatment is probably confounded/subject to self-selection
\end{itemize}
\item We can also estimate the Intention-to-Treat effect, the effect of the instrument itself
\begin{itemize}
\item But this will be \textbf{conservative}, i.e. less than the LATE estimate 
\end{itemize}
\end{itemize}
\end{itemize}
\end{frame}

\begin{frame}
\frametitle{Instrumental Variables}
\begin{itemize}
\item Critique (Deaton 2009):
\pause
\begin{itemize}
\item Our causal models need to represent a theory, not just be an arbitrary equation
\pause
\item If we use 'convenient' instruments, our causal effect and complier population are out of our control and might not be interesting
\pause
\item LATE causal estimates are not a good guide to policy effects
\pause
\item 'External' to our model is not the same as 'Exogenous', and we can't test exogeneity
\pause
\item Where the instrument is an arbitrary rule, there is often sorting as people re-adjust
\end{itemize}
\end{itemize}
\end{frame}

\section{Political Economy}

\begin{frame}
\frametitle{Social Accountability \& Information}
\begin{itemize}
\item Elections are not the only way in which elites are responsive to citizens
\pause
\item Citizens can also exert \textbf{direct} pressure to change decision-making
\begin{itemize}
\item Protests, lobbying
\item Checks and Balances through participatory institutions and the judiciary
\item The short-route of accountability: Client power in demanding public service improvements
\pause
\end{itemize}
\item Information \& Media also influence electoral accountability
\end{itemize}
\end{frame}

\begin{frame}
\frametitle{Social Accountability}
\begin{center}
\includegraphics[scale=0.45]{WDR.jpg}
\end{center}
\end{frame}

\begin{frame}
\frametitle{Social Accountability \& Information}
\begin{itemize}
\item Reinikka and Svensson (2005)
\begin{itemize}
\item 1995: Only 24\% of grants to schools arrive
\item 2002: 82\% of grants to schools arrive
\pause
\end{itemize}
\item This wasn't elite corruption, but diversions within the bureaucracy (centre -> district -> school)
\item What changed? A Government newspaper campaign to publicize grants
\end{itemize}
\end{frame}

\begin{frame}
\frametitle{Social Accountability \& Information}
\begin{itemize}
\item Reinikka and Svensson (2005)
\begin{itemize}
\item Aim to understand the impact of information on governance
\pause
\item What is the challenge to inference here? 
\pause
\item Information is not randomly assigned; eg. checks and balances on the bureaucracy may also be stronger in places where headteachers have more information
\end{itemize}
\end{itemize}
\end{frame}

\begin{frame}
\frametitle{Social Accountability \& Information}
\begin{itemize}
\item Reinikka and Svensson (2005)
\begin{itemize}
\item Schools close to Newspaper Seller -> + Information -> + \% Grant Received (-> + Enrollment, + Learning)
\pause
\end{itemize}
\item \textbf{Population:} \pause Ugandan Schools
\pause
\item \textbf{Sample:} \pause 218 Schools (mostly rural, stratified random sample)
\pause
\item \textbf{Treatment:} \pause New information on grants from newspapers

\item \textbf{Control:} \pause No new information on grants from newspapers
\pause
\item \textbf{Outcome:} \pause \% Grant Received (+Enrollment, Learning)
\pause
\item \textbf{Instrument:} \pause Distance to Newspaper Seller
\pause
\item \textbf{Treatment Assignment Mechanism:} \pause Messy! Influenced by confounders and instrument
\end{itemize}
\end{frame}

\begin{frame}
\frametitle{Social Accountability \& Information}
\begin{itemize}
\item Instrumental Variables Assumptions:
\pause
\begin{itemize}
\item \textbf{First-Stage:} \pause Closer to newspaper seller -> Headteacher knowledge of grant amount/timing
\begin{itemize}
\item Verifiable
\pause
\end{itemize}
\item \textbf{Exclusion Restriction:} \pause Distance to newspaper seller ONLY affects grant access and learning through information, not directly
\begin{itemize}
\item Unverifiable
\item But more likely when we include controls for distance to nearest bank, district headquarters etc.
\end{itemize}
\end{itemize}
\pause
\item They actually combine this with a difference-in-differences method to look at \textit{changes} in information and grant receipt over time.
\end{itemize}
\end{frame}

\begin{frame}
\frametitle{Social Accountability \& Information}
\begin{itemize}
\item Methodology:
\begin{itemize}
\item $Information_i = \alpha + \beta_0 Distance\_to\_Newspaper_i + \epsilon_i$
\item $Grant\_Received_i = \alpha + \beta_1 \hat{Information_i} + \epsilon_i$
\pause
\end{itemize}
\item Alternative:
\begin{itemize}
\item $Grant\_Received_i = \alpha + \beta_0 Distance\_to\_Newspaper_i + \epsilon_i$
\item $Enrolment = \alpha + \beta_1 \hat{Grant\_Received_i} + \epsilon_i$
\end{itemize}
\end{itemize}
\end{frame}

\begin{frame}
\frametitle{Social Accountability \& Information}
\begin{itemize}
\item Results:
\pause
\item A one standard deviation increase in information leads to 
\begin{itemize}
\item 44.2\% points more funding received
\item 297 students per school
\item 6\% better in exams
\end{itemize}
\end{itemize}
\end{frame}

\begin{frame}
\frametitle{Social Accountability \& Information}
\begin{itemize}
\item Critique?
\pause
\begin{itemize}
\item Distance to a newspaper seller is not exogenous - likely correlated with many factors
\pause
\item What type of information? Does it matter who communicates the information?
\begin{itemize}
\item Grant details also published by radio
\pause
\end{itemize}
\item Lots of other education system changes at the same time
\begin{itemize}
\item Enrollment doubled in 1997 when school became free
\item WB support conditional on better systems, transparency
\item Grants were also displayed on 90\% of school notice-boards
\pause
\end{itemize}
\item Where did these headteachers gain the political power to demand their grants?
\end{itemize}
\end{itemize}
\end{frame}

\begin{frame}
\frametitle{Social Accountability \& Information}
\begin{itemize}
\item Enikolopov et al (2011)
\begin{itemize}
\item Does independent media encourage voting for the opposition?
\pause
\item Russia: Does watching NTV encourage voting against pro-governemnt 'Unity'?
\pause
\end{itemize}
\item What is the inference problem?
\pause
\item People who watch NTV might be more anti-government in the first place
\item Or NTV may choose to broadcast in anti-government areas
\end{itemize}
\end{frame}

\begin{frame}
\frametitle{Social Accountability \& Information}
\begin{itemize}
\item Enikolopov et al (2011)
\begin{itemize}
\item Instrument watching NTV with the availability of the broadcast signal
\pause
\item \textbf{Population:} \pause All Russian voters
\pause
\item \textbf{Sample:} \pause All Russian voters (except Moscow, St. Petersburg and Chechnya) OR survey
\pause
\item \textbf{Treatment:} \pause Watching NTV
\pause
\item \textbf{Control:} \pause Not watching NTV
\pause
\item \textbf{Instrument:} \pause Availability of NTV broadcast signal
\pause
\item \textbf{Treatment Assignment Mechniams:} \pause Messy! Confounders, self-selection plus Instrument
\pause
\item \textbf{Outcome:} \pause Vote-share for each government/opposition party
\end{itemize}
\end{itemize}
\end{frame}

\begin{frame}
\frametitle{Social Accountability \& Information}
\begin{itemize}
\item Instrumental Variables Assumptions:
\begin{itemize}
\item \textbf{First Stage:} \pause Availability of signal clearly correlated with watching NTV
\pause
\item \textbf{Exclusion Restriction:} \pause Availability of the signal only affects voting through watching NTV
\end{itemize}
\end{itemize}
\end{frame}

\begin{frame}
\frametitle{Social Accountability \& Information}
\begin{itemize}
\item Exclusion Restriction Supporting Evidence:
\begin{itemize}
\item \textbf{History:} The transmitters were located for a Soviet education channel, not chosen by the opposition
\pause
\item \textbf{Controls:} Transmitters are correlated with socioeconomic characteristics, but we can control for this (urban, population, wage)
\pause
\item \textbf{Placebo:} If the instrument only operates through treatment, it should have no effect when treatment is impossible, eg. in 1995
\end{itemize}
\end{itemize}
\end{frame}

\begin{frame}
\frametitle{Social Accountability \& Information}
\begin{itemize}
\item Estimate signal availability using Irregular Terrain Model and transmitter power/frequency
\end{itemize}
\end{frame}

\begin{frame}
\frametitle{Social Accountability}
\begin{center}
\includegraphics[scale=0.55, page=3, trim=3cm 0 0 4cm]{EZ_map.pdf}
\end{center}
\end{frame}

\begin{frame}
\frametitle{Social Accountability}
\begin{itemize}
\item \textbf{Aggregate Level Data (effect of NTV availability):}
\begin{itemize}
\item $NTV\_available_i =  \alpha + \beta_0 + Signal\_Strength_i + \epsilon_i$
\item $vote_i = \alpha + \beta_1 \hat{NTV\_available_i} + \beta_2 X_i + Region\_FEs + \epsilon_i$
\end{itemize}
\pause
\item \textbf{Individual Level Data (effect of watching NTV):}
\begin{itemize}
\item $Watch\_NTV_i = \alpha + \beta_0 Signal\_Strength_i + \epsilon_i$
\item $vote_i = \alpha + \beta_1 \hat{Watch\_NTV_i} + \beta_2 X_i + Region\_FEs + \epsilon_i$
\end{itemize}
\end{itemize}
\end{frame}

\begin{frame}
\frametitle{Social Accountability \& Information}
\begin{itemize}
\item Results:
\pause
\begin{itemize}
\item NTV broadcast availability reduces pro-government 'Unity' voting by 8.9\% points (official results)
\item NTV broadcast availability reduces turnout by 3.8\% points (official results)
\item Watching NTV broadcast reduces pro-government 'Unity' voting by 26\% (survey results)
\end{itemize}
\end{itemize}
\end{frame}

\begin{frame}
\frametitle{Social Accountability \& Information}
\begin{itemize}
\item Acemoglu \& Robinson (2001)
\begin{itemize}
\item Non-electoral institutions (property rights, checks and balances) drive accountability and growth
\pause
\item Institutions depend on powerful elites, esp. colonial settlers
\pause
\item Extractive vs. Settler Institutions
\pause
\item Colonial Strategy -> Institutions -> Growth
\end{itemize}
\item What is the inferential problem here?
\end{itemize}
\end{frame}

\begin{frame}
\frametitle{Social Accountability \& Information}
\begin{itemize}
\item Acemoglu \& Robinson (2001)
\begin{itemize}
\item Instrument Institutions with settler mortality rates
\end{itemize}
\item \textbf{Population:} \pause Ex-colonies
\pause
\item \textbf{Sample:} \pause Ex-colonies
\pause
\item \textbf{Treatment:} \pause Settler institutions (measured by 'risk of expropriation' index 1985-95)
\pause
\item \textbf{Control:} \pause Extractive institutions
\pause
\item \textbf{Instrument:} \pause Settler (soldier...) mortality rates
\pause
\item \textbf{Treatment Assignment Mechniams:} \pause Messy! Confounders plus Instrument
\pause
\item \textbf{Outcome:} \pause Growth rates in 1995
\end{itemize}
\end{frame}

\begin{frame}
\frametitle{Social Accountability \& Information}
\begin{itemize}
\item Instrumental Variables Assumptions:
\pause
\begin{itemize}
\item \textbf{First Stage:} \pause Settler Mortality explains Current Institutions
\pause
\item \textbf{Exclusion Restriction:} \pause Settler Mortality only affects growth through institutions
\end{itemize}
\end{itemize}
\end{frame}

\begin{frame}
\frametitle{Social Accountability \& Information}
\begin{itemize}
\item Exclusion Restriction Supporting Evidence:
\begin{itemize}
\item Disease environment doesn't affect human capital/growth directly because locals have adapted
\pause
\item Control for possible correlates - geography, climate, etc.
\end{itemize}
\end{itemize}
\end{frame}

\begin{frame}
\frametitle{Social Accountability \& Information}
\begin{itemize}
\item Methodology:
\begin{itemize}
\item $Institutions_i = \alpha + \beta_0 Settler\_Mortality_i + \epsilon_i$
\item $Growth_i = \alpha + \beta_1 \hat{Institutions_i} + \epsilon_i$
\end{itemize}
\end{itemize}
\end{frame}

\begin{frame}
\frametitle{Social Accountability \& Information}
\begin{itemize}
\item Results: Improving Nigeria's institutions to Chile's level would raise GDP 7-fold
\end{itemize}
\end{frame}

\begin{frame}
\frametitle{Social Accountability \& Information}
\begin{itemize}
\item 'Social' Accountability can dramatically affect public services, voting behaviour and growth
\begin{itemize}
\item Client Power to demand more from government
\item Exposure to information/Media
\item Checks and Balances on expropriation
\end{itemize}
\end{itemize}
\end{frame}

\end{document}


%setwd('C:\\Users\\Jonny\\Google Drive\\Academic\\USP\\Class\\Week 1 - Intro\\Lecture Slides')
%knitr::knit("Slides_Wk1_intro_5.Rnw")
