% Font options: 10pm, 11pt, 12pt
% Align headings left instead of center: nocenter
\documentclass[xcolor=x11names,compress]{beamer}\usepackage[]{graphicx}\usepackage[]{color}
%% maxwidth is the original width if it is less than linewidth
%% otherwise use linewidth (to make sure the graphics do not exceed the margin)
\makeatletter
\def\maxwidth{ %
  \ifdim\Gin@nat@width>\linewidth
    \linewidth
  \else
    \Gin@nat@width
  \fi
}
\makeatother

\definecolor{fgcolor}{rgb}{0.345, 0.345, 0.345}
\newcommand{\hlnum}[1]{\textcolor[rgb]{0.686,0.059,0.569}{#1}}%
\newcommand{\hlstr}[1]{\textcolor[rgb]{0.192,0.494,0.8}{#1}}%
\newcommand{\hlcom}[1]{\textcolor[rgb]{0.678,0.584,0.686}{\textit{#1}}}%
\newcommand{\hlopt}[1]{\textcolor[rgb]{0,0,0}{#1}}%
\newcommand{\hlstd}[1]{\textcolor[rgb]{0.345,0.345,0.345}{#1}}%
\newcommand{\hlkwa}[1]{\textcolor[rgb]{0.161,0.373,0.58}{\textbf{#1}}}%
\newcommand{\hlkwb}[1]{\textcolor[rgb]{0.69,0.353,0.396}{#1}}%
\newcommand{\hlkwc}[1]{\textcolor[rgb]{0.333,0.667,0.333}{#1}}%
\newcommand{\hlkwd}[1]{\textcolor[rgb]{0.737,0.353,0.396}{\textbf{#1}}}%
\let\hlipl\hlkwb

\usepackage{framed}
\makeatletter
\newenvironment{kframe}{%
 \def\at@end@of@kframe{}%
 \ifinner\ifhmode%
  \def\at@end@of@kframe{\end{minipage}}%
  \begin{minipage}{\columnwidth}%
 \fi\fi%
 \def\FrameCommand##1{\hskip\@totalleftmargin \hskip-\fboxsep
 \colorbox{shadecolor}{##1}\hskip-\fboxsep
     % There is no \\@totalrightmargin, so:
     \hskip-\linewidth \hskip-\@totalleftmargin \hskip\columnwidth}%
 \MakeFramed {\advance\hsize-\width
   \@totalleftmargin\z@ \linewidth\hsize
   \@setminipage}}%
 {\par\unskip\endMakeFramed%
 \at@end@of@kframe}
\makeatother

\definecolor{shadecolor}{rgb}{.97, .97, .97}
\definecolor{messagecolor}{rgb}{0, 0, 0}
\definecolor{warningcolor}{rgb}{1, 0, 1}
\definecolor{errorcolor}{rgb}{1, 0, 0}
\newenvironment{knitrout}{}{} % an empty environment to be redefined in TeX

\usepackage{alltt}
%\documentclass[xcolor=x11names,compress,handout]{beamer}
\usepackage[]{graphicx}
\usepackage[]{color}
\usepackage{booktabs}
\usepackage{hyperref}
\usepackage{tikz}
\usepackage{multirow}
\usepackage{multicol}
\usepackage{dcolumn}
\usepackage{bigstrut}
\usepackage{amsmath} 
\usepackage{xcolor,colortbl}
\usepackage{amssymb}
%\newcommand{\done}{\cellcolor{teal}#1}

%% Beamer Layout %%%%%%%%%%%%%%%%%%%%%%%%%%%%%%%%%%
\useoutertheme[subsection=false,shadow]{miniframes}
\useinnertheme{default}
\usefonttheme{serif}
\usepackage{Arev}
\usepackage{pdfpages}

\setbeamerfont{title like}{shape=\scshape}
\setbeamerfont{frametitle}{shape=\scshape, size=\normalsize}

\definecolor{dkblue}{RGB}{0,0,102}

\setbeamercolor*{lower separation line head}{bg=dkblue} 
\setbeamercolor*{normal text}{fg=black,bg=white} 
\setbeamercolor*{alerted text}{fg=red} 
\setbeamercolor*{example text}{fg=black} 
\setbeamercolor*{structure}{fg=black} 
 
\setbeamercolor*{palette tertiary}{fg=black,bg=black!10} 
\setbeamercolor*{palette quaternary}{fg=black,bg=black!10} 

\renewcommand{\(}{\begin{columns}}
\renewcommand{\)}{\end{columns}}
\newcommand{\<}[1]{\begin{column}{#1}}
\renewcommand{\>}{\end{column}}

\setbeamertemplate{navigation symbols}{} 
\setbeamertemplate{footline}[frame number]
\setbeamertemplate{caption}{\raggedright\insertcaption\par}

\setbeamersize{text margin left=5pt,text margin right=5pt}

%%%%%%%%%%%%%%%%%%%%%%%%%%%%%%%%%%%%%%%%%%%%%%%%%%




\title{FLS 6441 - Methods III: Explanation and Causation}
\subtitle{Week 1 - Review}
\author{Jonathan Phillips}
\date{February 2019}
\IfFileExists{upquote.sty}{\usepackage{upquote}}{}
\begin{document}

\frame{\titlepage}

\section{Introduction}

\begin{frame}
\frametitle{Course Objectives}
\begin{enumerate}
\item temp
\end{enumerate}
\end{frame}

\section{Probability}

\begin{frame}
\frametitle{Data}
\begin{enumerate}
\item We work with variables, which VARY!
\end{enumerate}
\begin{multicols}{2}
% latex table generated in R 3.5.2 by xtable 1.8-3 package
% Thu Feb 28 08:57:39 2019
\begin{table}[ht]
\centering
\begin{tabular}{rr}
  \hline
 & Variable \\ 
  \hline
1 & -1.21 \\ 
  2 & 0.73 \\ 
  3 & -0.51 \\ 
  4 & -1.41 \\ 
  5 & 0.62 \\ 
  6 & 0.33 \\ 
  7 & -0.66 \\ 
  8 & 0.31 \\ 
  9 & -0.12 \\ 
  10 & 0.16 \\ 
   \hline
\end{tabular}
\end{table}

\columnbreak
\begin{knitrout}
\definecolor{shadecolor}{rgb}{0.969, 0.969, 0.969}\color{fgcolor}
\includegraphics[width=\maxwidth]{figure/var2-1} 

\end{knitrout}
\end{multicols}
\end{frame}

\section{What does Regression do?}

\begin{frame}
\frametitle{Regression}
\begin{itemize}
\item Regression is a \textbf{Conditional Expectation Function}
\pause
\item Conditional on $x$, what is our expectation (mean value) of $y$?
\pause
\item $E(y|x)$
\pause
\item When age is 20 ($x=40$), the average salary is R1.000 ($y=1.000$)
\item When age is 40 ($x=40$), the average salary is R2.000 ($y=2.000$)
\end{itemize}
\end{frame}

\begin{frame}
\frametitle{Regression}
\begin{itemize}
\item Regression is a \textbf{Conditional Expectation Function}
\end{itemize}
%Graph of CEF
\end{frame}

\begin{knitrout}
\definecolor{shadecolor}{rgb}{0.969, 0.969, 0.969}\color{fgcolor}\begin{kframe}
\begin{verbatim}
## How to cite this model in Zelig:
##   R Core Team. 2007.
##   ls: Least Squares Regression for Continuous Dependent Variables
##   in Christine Choirat, Christopher Gandrud, James Honaker, Kosuke Imai, Gary King, and Olivia Lau,
##   "Zelig: Everyone's Statistical Software," http://zeligproject.org/
## 
## % Table created by stargazer v.5.2.2 by Marek Hlavac, Harvard University. E-mail: hlavac at fas.harvard.edu
## % Date and time: Thu, Feb 28, 2019 - 8:57:42 AM
## \begin{table}[!htbp] \centering 
##   \caption{} 
##   \label{} 
## \begin{tabular}{@{\extracolsep{5pt}}lc} 
## \\[-1.8ex]\hline 
## \hline \\[-1.8ex] 
##  & \multicolumn{1}{c}{\textit{Dependent variable:}} \\ 
## \cline{2-2} 
## \\[-1.8ex] & y \\ 
## \hline \\[-1.8ex] 
##  x & 0.758$^{***}$ \\ 
##   & (0.066) \\ 
##   & \\ 
##  Constant & $-$0.000 \\ 
##   & (0.066) \\ 
##   & \\ 
## \hline \\[-1.8ex] 
## Observations & 100 \\ 
## R$^{2}$ & 0.575 \\ 
## Adjusted R$^{2}$ & 0.571 \\ 
## Residual Std. Error & 0.655 (df = 98) \\ 
## F Statistic & 132.700$^{***}$ (df = 1; 98) \\ 
## \hline 
## \hline \\[-1.8ex] 
## \textit{Note:}  & \multicolumn{1}{r}{$^{*}$p$<$0.1; $^{**}$p$<$0.05; $^{***}$p$<$0.01} \\ 
## \end{tabular} 
## \end{table}
\end{verbatim}
\end{kframe}
\end{knitrout}


\begin{frame}
\frametitle{Regression}
\begin{itemize}
\item Regression with two variables is very similar to calculating correlation
\pause
\item $\hat{\beta}=cor(x,y) * \frac{\sigma_Y}{\sigma_X}$
\pause
\item It's \textit{identical} if we standardize both variables first ($\frac{(x-\bar{x})}{\sigma_x}$)
\end{itemize}
\begin{multicols}{2}
\begin{knitrout}
\definecolor{shadecolor}{rgb}{0.969, 0.969, 0.969}\color{fgcolor}
\includegraphics[width=\maxwidth]{figure/corr_regn_fig1-1} 

\end{knitrout}
\columnbreak
\end{multicols}
\end{frame}

\begin{frame}
\frametitle{Regression}
\begin{itemize}
\item Regression with two variables is very similar to calculating correlation:
\item $\hat{\beta}=cor(x,y) * \frac{\sigma_Y}{\sigma_X}$
\item It's \textit{identical} if we standardize both variables first ($\frac{(x-\bar{x})}{\sigma_x}$)
\end{itemize}
\begin{multicols}{2}
\begin{knitrout}
\definecolor{shadecolor}{rgb}{0.969, 0.969, 0.969}\color{fgcolor}
\includegraphics[width=\maxwidth]{figure/corr_regn_fig2-1} 

\end{knitrout}
\columnbreak
Correlation is 0.758
\end{multicols}
\end{frame}

\begin{frame}
\frametitle{Regression}
\begin{itemize}
\item Regression with two variables is very similar to calculating correlation
\item $\hat{\beta}=cor(x,y) * \frac{\sigma_Y}{\sigma_X}$
\item It's \textit{identical} if we standardize both variables first ($\frac{(x-\bar{x})}{\sigma_x}$)
\end{itemize}
\begin{multicols}{2}
\begin{knitrout}
\definecolor{shadecolor}{rgb}{0.969, 0.969, 0.969}\color{fgcolor}
\includegraphics[width=\maxwidth]{figure/corr_regn_fig4-1} 

\end{knitrout}
\columnbreak
The regression result is:

\begin{table}[!htbp] \centering 
  \caption{} 
  \label{} 
\begin{tabular}{@{\extracolsep{5pt}}lc} 
\\[-1.8ex]\hline 
\hline \\[-1.8ex] 
 & \multicolumn{1}{c}{\textit{Dependent variable:}} \\ 
\cline{2-2} 
\\[-1.8ex] & y \\ 
\hline \\[-1.8ex] 
 x & 0.758$^{***}$ \\ 
  & (0.066) \\ 
  & \\ 
 Constant & $-$0.000 \\ 
  & (0.066) \\ 
  & \\ 
\hline \\[-1.8ex] 
Observations & 100 \\ 
R$^{2}$ & 0.575 \\ 
Adjusted R$^{2}$ & 0.571 \\ 
Residual Std. Error & 0.655 (df = 98) \\ 
F Statistic & 132.700$^{***}$ (df = 1; 98) \\ 
\hline 
\hline \\[-1.8ex] 
\textit{Note:}  & \multicolumn{1}{r}{$^{*}$p$<$0.1; $^{**}$p$<$0.05; $^{***}$p$<$0.01} \\ 
\end{tabular} 
\end{table} 

\end{multicols}
\end{frame}

\begin{frame}
\frametitle{Regression}
\begin{itemize}
\item Regression with two variables is very similar to calculating correlation
\item $\hat{\beta}=cor(x,y) * \frac{\sigma_Y}{\sigma_X}$
\item It's \textit{identical} if we standardize both variables first ($\frac{(x-\bar{x})}{\sigma_x}$)
\end{itemize}
\begin{multicols}{2}
\begin{knitrout}
\definecolor{shadecolor}{rgb}{0.969, 0.969, 0.969}\color{fgcolor}
\includegraphics[width=\maxwidth]{figure/corr_regn_fig3-1} 

\end{knitrout}
\columnbreak
The regression result is:

\begin{table}[!htbp] \centering 
  \caption{} 
  \label{} 
\begin{tabular}{@{\extracolsep{5pt}}lc} 
\\[-1.8ex]\hline 
\hline \\[-1.8ex] 
 & \multicolumn{1}{c}{\textit{Dependent variable:}} \\ 
\cline{2-2} 
\\[-1.8ex] & y \\ 
\hline \\[-1.8ex] 
 x & 1.012$^{***}$ \\ 
  & (0.088) \\ 
  & \\ 
 Constant & $-$0.037 \\ 
  & (0.190) \\ 
  & \\ 
\hline \\[-1.8ex] 
Observations & 100 \\ 
R$^{2}$ & 0.575 \\ 
Adjusted R$^{2}$ & 0.571 \\ 
Residual Std. Error & 0.911 (df = 98) \\ 
F Statistic & 132.700$^{***}$ (df = 1; 98) \\ 
\hline 
\hline \\[-1.8ex] 
\textit{Note:}  & \multicolumn{1}{r}{$^{*}$p$<$0.1; $^{**}$p$<$0.05; $^{***}$p$<$0.01} \\ 
\end{tabular} 
\end{table} 

\end{multicols}
\end{frame}

\section{Guide to Designing Regressions}

\begin{frame}
\frametitle{Regression Guide}
\begin{enumerate}
\item \textbf{Choose variables and measures:} To test a specific hypothesis
\item \textbf{Choose a Model/Link Function:} Should match the data type of your outcome variable
\item \textbf{Choose Covariates:} To match your strategy of inference
\item \textbf{Choose Fixed Effects:} To focus on a specific level of variation
\item \textbf{Choose Standard Error Structure:} To match known dependencies/clustering in the data
\item \textbf{Interpret the coefficients:} Depending on the type/scale of the explanatory variable
\end{enumerate}
\end{frame}

\begin{frame}
\frametitle{Regression Models}
The Regression Model reflects the data type of the outcome variable:
\begin{itemize}
\item Continuous -> Ordinary Least Squares  
\begin{knitrout}
\definecolor{shadecolor}{rgb}{0.969, 0.969, 0.969}\color{fgcolor}\begin{kframe}
\begin{alltt}
\hlkwd{zelig}\hlstd{(Y} \hlopt{~} \hlstd{X,}\hlkwc{data}\hlstd{=d,}\hlkwc{model}\hlstd{=}\hlstr{"ls"}\hlstd{)}
\end{alltt}
\end{kframe}
\end{knitrout}
\item Binary -> Logit  
\begin{knitrout}
\definecolor{shadecolor}{rgb}{0.969, 0.969, 0.969}\color{fgcolor}\begin{kframe}
\begin{alltt}
\hlkwd{zelig}\hlstd{(Y} \hlopt{~} \hlstd{X,}\hlkwc{data}\hlstd{=d,}\hlkwc{model}\hlstd{=}\hlstr{"logit"}\hlstd{)}
\end{alltt}
\end{kframe}
\end{knitrout}
\item Unordered categories -> Multinomial logit  
\begin{knitrout}
\definecolor{shadecolor}{rgb}{0.969, 0.969, 0.969}\color{fgcolor}\begin{kframe}
\begin{alltt}
\hlkwd{zelig}\hlstd{(Y} \hlopt{~} \hlstd{X,}\hlkwc{data}\hlstd{=d,}\hlkwc{model}\hlstd{=}\hlstr{"mlogit"}\hlstd{)}
\end{alltt}
\end{kframe}
\end{knitrout}
\item Ordered categories -> Ordered logit  
\begin{knitrout}
\definecolor{shadecolor}{rgb}{0.969, 0.969, 0.969}\color{fgcolor}\begin{kframe}
\begin{alltt}
\hlkwd{zelig}\hlstd{(Y} \hlopt{~} \hlstd{X,}\hlkwc{data}\hlstd{=d,}\hlkwc{model}\hlstd{=}\hlstr{"ologit"}\hlstd{)}
\end{alltt}
\end{kframe}
\end{knitrout}
\item Count -> Poisson  
\begin{knitrout}
\definecolor{shadecolor}{rgb}{0.969, 0.969, 0.969}\color{fgcolor}\begin{kframe}
\begin{alltt}
\hlkwd{zelig}\hlstd{(Y} \hlopt{~} \hlstd{X,}\hlkwc{data}\hlstd{=d,}\hlkwc{model}\hlstd{=}\hlstr{"poisson"}\hlstd{)}
\end{alltt}
\end{kframe}
\end{knitrout}
\end{itemize}
\end{frame}

\begin{frame}
\frametitle{Interpreting Regression Results}
\begin{itemize}
\item Difficult! It depends on the scale of the explanatory variable, scale of the outcome, the regression model we used, and the presence of any interaction
\item Basic OLS:
\begin{itemize}
\item 1 [unit of explanatory variable] change in the explanatory variable is associated with a $\beta$ [unit of outcome variable] change in the outcome
\end{itemize}
\end{itemize}
\end{frame}

\begin{frame}
\frametitle{Predictions from Regressions}
\begin{itemize}
\item temp
\end{itemize}
\end{frame}

%PVs for OLS, for logit, FDs
%PVs vs. EVs

\section{What does Regression NOT do?}



\begin{frame}
\frametitle{Omitted Variable Bias}
\begin{knitrout}
\definecolor{shadecolor}{rgb}{0.969, 0.969, 0.969}\color{fgcolor}
\includegraphics[width=\maxwidth]{figure/confound3b-1} 

\end{knitrout}
\end{frame}

\begin{frame}
\frametitle{Omitted Variable Bias}
\begin{knitrout}
\definecolor{shadecolor}{rgb}{0.969, 0.969, 0.969}\color{fgcolor}
\includegraphics[width=\maxwidth]{figure/confound3c-1} 

\end{knitrout}
\end{frame}

\begin{frame}
\frametitle{Omitted Variable Bias}
\begin{knitrout}
\definecolor{shadecolor}{rgb}{0.969, 0.969, 0.969}\color{fgcolor}
\includegraphics[width=\maxwidth]{figure/confound2-1} 

\end{knitrout}
\end{frame}


\begin{frame}
\frametitle{Omitted Variable Bias}
\begin{knitrout}
\definecolor{shadecolor}{rgb}{0.969, 0.969, 0.969}\color{fgcolor}
\includegraphics[width=\maxwidth]{figure/confound3-1} 

\end{knitrout}
\end{frame}

% Overlap/Functional form error
% Measurement Error
% Omitted Variable Bias
% Reverse Causation/Endogeneity
% Self-Selection Bias
% Data Selection Bias

% Stress - in prep for week 2 - that regression only buys you (conditional) correlation

%Chenage all examples to age-gender-income

\end{document}


%setwd('C:\\Users\\Jonny\\Google Drive\\Academic\\USP\\Class\\Week 1 - Intro\\Lecture Slides')
%knitr::knit("Slides_Wk1_intro_5.Rnw")
